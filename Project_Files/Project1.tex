\documentclass[11pt]{article}
\textwidth 7.5in
\textheight 10in
\oddsidemargin -.5in
\evensidemargin -.5in
\topmargin -1in    
\usepackage{graphics}        
\usepackage{graphicx}
\usepackage{amsmath,amsthm,amscd,amssymb}
\usepackage{latexsym}
\usepackage{upref}
\begin{document}
\newcommand{\dsp}{\displaystyle}
\newcommand{\ihat}{{\bf{i}}}
\newcommand{\jhat}{{\bf{j}}}
\newcommand{\khat}{{\bf{k}}}
\newcommand{\Fhat}{{\bf{F}}}
%\pagestyle{headings}
%\markright{MATH 1120 Lab Worksheet \#1}

\thispagestyle{empty}

\noindent
\sffamily
\begin{center}
\rule{7.5in}{2pt}

\vspace{.2in}

\begin{tabular}{p{4in}p{3.5in}}
MATH 3025

\vspace{.2in}

Computer Lab \#1
& 
NAME:  $_{\rule{2.5in}{1pt}}$

\vspace{.2in}

Due Monday February 5th 2023 by noon.
\\*[.2in]
\end{tabular}
\rule{7.5in}{2pt}

\vspace{.1in}
All work is to be done in a notebook with both code and written comments.  Please utilize the notebook structure to type comments in their own text boxes. \\
 Please refer to the blackboard and GitHub for commands and examples. \\
 Submissions must be made electronically on blackboard. \\
   You will be graded on the output that I am able to generate from your commands.\\
%\bf Please delete all output before submitting.
\end{center}
%\begin{tabular}{*{26}{|c}|}\hline
%A&B&C&D&E&F&G&H&I&J&K&L&M&N&O&P&Q&R&S&T&U&V&W&X&Y&Z\\ \hline
%1&2&3&4&5&6&7&8&9&10&11&12&13&14&15&16&17&18&19&20&21&22&23&24&25&26\\ \hline
%\end{tabular}

\begin{enumerate}
\item Let the last digit of your student number be $a$ (if it happens to be a zero use $a=10$).  Consider the function 
\[f(x)=\frac1{\sqrt{ax+1}}.\]
\begin{enumerate}
\item Graph the function from 0 to 1.
\item Make an estimate of the area under the curve based on the graph.  There should be complete sentences explaining your estimate.
\item Calculate the area under the curve.
\end{enumerate}
\item Consider the functions, where $a$ is as defined above,
\[
y= e^{x}\quad\quad\text{and}\quad\quad y=\sqrt{a x}+1.
\]
\begin{enumerate}
\item Graph both functions on the same coordinate axis.
\item Find the points of intersection using the numerical solver and inspection of the graph.
\item Calculate the area between the two curves.
\item Calculate the volume of the solid obtained by rotating about the $y$-axis.
\end{enumerate}
\item Integration by parts can be used to prove the following reduction formula.
\[
\int \left(\ln x\right)^n dx=x\left(\ln x\right)^n-n\int\left(\ln x \right)^{n-1} dx
\]
\begin{enumerate}
\item Find the integral on the left-hand-side when $n=a$.
\item Explain why the reduction formula is preferred to the above output.  (If you had a=1 or 2 you might try a different number)
\item Prove the reduction formula by having programming language take the derivative of the right-hand-side.
\end{enumerate}
\end{enumerate}
\end{document}





