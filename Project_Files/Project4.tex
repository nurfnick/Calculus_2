\documentclass[11pt]{article}
\textwidth 7.5in
\textheight 10in
\oddsidemargin -.5in
\evensidemargin -.5in
\topmargin -1in    
\usepackage{graphics}        
\usepackage{graphicx}
\usepackage{amsmath,amsthm,amscd,amssymb}
\usepackage{latexsym}
\usepackage{upref}
\begin{document}
\newcommand{\dsp}{\displaystyle}
\newcommand{\ihat}{{\bf{i}}}
\newcommand{\jhat}{{\bf{j}}}
\newcommand{\khat}{{\bf{k}}}
\newcommand{\Fhat}{{\bf{F}}}
%\pagestyle{headings}
%\markright{MATH 1120 Lab Worksheet \#1}

\thispagestyle{empty}

\noindent
\sffamily
\begin{center}
\rule{7.5in}{2pt}

\vspace{.2in}

\begin{tabular}{p{4in}p{3.5in}}
MATH 3025

\vspace{.2in}

Mathematica Lab \#4
& 
NAME:  $_{\rule{2.5in}{1pt}}$

\vspace{.2in}

Due Monday May 1st 2023 by 11:30 A.M.
\\*[.2in]
\end{tabular}
\rule{7.5in}{2pt}

\vspace{.1in}
All work is to be done in a programming notebook. \\
 Please refer to the blackboard site for commands and examples. \\
 Submissions must be made electronically to njacob@ecok.edu. \\
   You will be graded on the output that I am able to generate from your commands.\\
%\bf Please delete all output before submitting.
\end{center}
\begin{tabular}{*{26}{|c}|}\hline
A&B&C&D&E&F&G&H&I&J&K&L&M&N&O&P&Q&R&S&T&U&V&W&X&Y&Z\\ \hline
1&2&3&4&5&6&7&8&9&10&11&12&13&14&15&16&17&18&19&20&21&22&23&24&25&26\\ \hline
\end{tabular}

\begin{enumerate}
\item Let $a$ be the value corresponding to the first letter of your last (family) name.  Consider the Bessel function of the first kind
\[
J_a(x)
=
\sum_{n=0}^\infty
\frac{\left(-1\right)^n x^{2n+a}}
{n!\left(n+a\right)!\  2^{2n+a}}
\]
\begin{enumerate}
\item Find the interval of convergence.
\item Graph the partial sums for $0\leq n\leq b$ for $0\leq x\leq10$.  Where $b$ is 1, 10, and 50.
\item Graph the partial sums and the full function using the besselj(a,x) command all on the same coordinate axis.
\item Explain in words how well the partial sums approximate the full function.
\end{enumerate}
\item Consider the Lissajous figure.  Use $a$ as defined above and $b$ as the initial of your first name.  Your $t$ should vary between zero and $2\pi$.
\[
\left\{
\begin{array}{l}
x=a\sin 3t\\
y=b\cos t
\end{array}
\right.
\]
\begin{enumerate}
\item Use plot\_parametric command graph the curve.
\item Find the equation of the tangent when $t=\frac\pi{12}$.  Graph it together with the curve.  
\item Find the area under the curve.
\item Find the arc length of the curve.
\end{enumerate}
\end{enumerate}
\end{document}





