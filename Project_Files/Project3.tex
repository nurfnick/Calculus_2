\documentclass[11pt]{article}
\textwidth 7.5in
\textheight 10in
\oddsidemargin -.5in
\evensidemargin -.5in
\topmargin -1in    
\usepackage{graphics}        
\usepackage{graphicx}
\usepackage{amsmath,amsthm,amscd,amssymb}
\usepackage{latexsym}
\usepackage{upref}
\begin{document}
\newcommand{\dsp}{\displaystyle}
\newcommand{\ihat}{{\bf{i}}}
\newcommand{\jhat}{{\bf{j}}}
\newcommand{\khat}{{\bf{k}}}
\newcommand{\Fhat}{{\bf{F}}}
%\pagestyle{headings}
%\markright{MATH 1120 Lab Worksheet \#1}

\thispagestyle{empty}

\noindent
\sffamily
\begin{center}
\rule{7.5in}{2pt}

\vspace{.2in}

\begin{tabular}{p{4in}p{3.5in}}
MATH 3025

\vspace{.2in}

Mathematica Lab \#3
& 
NAME:  $_{\rule{2.5in}{1pt}}$

\vspace{.2in}

Due Monday April 3rd 2023 by 11:59 P.M.
\\*[.2in]
\end{tabular}
\rule{7.5in}{2pt}

\vspace{.1in}
All work is to be done in a programming notebook. \\
 Please refer to the blackboard site for commands and examples. \\
 %Submissions must be made electronically to njacob@ecok.edu. \\
   You will be graded on the output that I am able to generate from your commands.\\

\end{center}
\begin{tabular}{*{26}{|c}|}\hline
A&B&C&D&E&F&G&H&I&J&K&L&M&N&O&P&Q&R&S&T&U&V&W&X&Y&Z\\ \hline
27&2&3&4&5&6&7&8&9&10&11&12&13&14&15&16&17&18&19&20&21&22&23&24&25&26\\ \hline
\end{tabular}

\begin{enumerate}
\item
Consider the function $f(x)=\frac1x$ for $x>1$.  Revolve this function about the $x$-axis creates what is referred to as Gabriel's (or Torricelli's) Horn.  A picture is provided below.

\begin{center}
\includegraphics[scale=.4]{GabrielsHorn}
\end{center}
\begin{enumerate}
\item Calculate the volume of the horn for $1<x<\infty$.
\item Calculate the anti-derivative that would represent the surface area.\label{b}
\item Use part \ref{b} to compute the surface area for $1<x<\infty$ by taking the limit as $t\to\infty$.
\item Explain in words why this is a paradox.  It is best to talk in terms of paint.
\end{enumerate}
\item Consider the function $g(x)=\dfrac {x^5}{x!}$.  Let $a$ be the number corresponding to the first initial of your family (last) name.
\begin{enumerate}
\item Graph the function for $0\leq x\leq20$.  Note the plot command will not work since $g(x)$ is not defined for $n<x<n+1$ for all $n\in\mathbb{N}$.  See template for project 3 for a suggestion on which package to use.
\item Guess the limit of $g(x)$ as $x$ approaches infinity.
\item Find the smallest value $N$ that correspond to $\epsilon=\frac 1{a^3}$ in the precise (formal) definition of the limit.  See definition 2 on page 692 of your book or the wikipedia page on formal definition of the limit of a sequence.
\end{enumerate}
\item Consider the series.
\[
\sum_{n=1}^\infty\frac{\left(\ln n\right)^2}{n^2}
\]
\begin{enumerate}
\item Show that the series is convergent by computing its sum,  $S$.
\item Let $\displaystyle S_a=\sum_{n=1}^a\frac{\left(\ln n\right)^2}{n^2}$, find the error in $S\approx S_a$.
\item Find the error in $S\approx S_{a+100}$ and $S\approx S_{a+1000}$.

\end{enumerate}
\end{enumerate}
\end{document}





